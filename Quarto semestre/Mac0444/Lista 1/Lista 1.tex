% https://www.ctan.org/tex-archive/macros/latex/contrib/logicproof?lang=en
% https://en.wikibooks.org/wiki/LaTeX/Installing_Extra_Packages
% latex logicproof.ins
% latex logicproof.dtx
% mv logicproof.sty ~/texmf/tex/latex/logicproof/
% cd ~
% texhash texmf


\documentclass[12pt, a4paper]{article} %mostra o tipo do documento
\setlength{\topmargin}{-.5in}
\setlength{\textheight}{9in}
\setlength{\textwidth}{6.3in}
\setlength{\oddsidemargin}{-.125in}
\setlength{\evensidemargin}{-.125in}
\usepackage[brazil]{babel} % permite escrever em português
\usepackage[utf8]{inputenc}
\usepackage[T1]{fontenc} % define a fonte das letras
\usepackage{amsmath, amssymb, amsthm, amsfonts} % permite fazer textos matemáticos
\usepackage{float} % permite mover tabelas e figuras para qualquer ponto da página
\usepackage{graphicx} % permite colocar imagens no documento
\usepackage{color} % permite colorir o texto
\usepackage{listings}
\usepackage{logicproof}

\begin{document}
\begin{logicproof}{2}
(p\lor q)\lor r & premise\\
\begin{subproof}
(p\lor q) & assumption\\
\begin{subproof}
p & assumption\\
p\lor (q\lor r) & $\lor\mathrm{i}_1$, 3
\end{subproof}
\begin{subproof}
q & assumption\\
q\lor r & $\lor\mathrm{i}_1$, 5\\
p\lor (q\lor r) & $\lor\mathrm{i}_2$, 6
\end{subproof}
p\lor (q\lor r) & $\lor$e, 2, 3--4, 5--7
\end{subproof}
\begin{subproof}
r & assumption\\
q\lor r & $\lor\mathrm{i}_2$, 9\\
p\lor (q\lor r) & $\lor\mathrm{i}_2$, 10
\end{subproof}
p\lor (q\lor r) & $\lor$e, 1, 2--8, 9--11
\end{logicproof}
\end{document}