\documentclass[12pt, a4paper]{article} %mostra o tipo do documento
\setlength{\topmargin}{-.5in}
\setlength{\textheight}{9in}
\setlength{\textwidth}{6.3in}
\setlength{\oddsidemargin}{-.125in}
\setlength{\evensidemargin}{-.125in}
\usepackage[brazil]{babel} %permite escrever em português
\usepackage[utf8]{inputenc}
\usepackage[T1]{fontenc} %define a fonte das letras
\usepackage{amsmath, amssymb, amsthm, amsfonts} %permite fazer textos matemáticos
\usepackage{float} % permite mover tabelas e figuras para qualquer ponto da página
\usepackage{graphicx} %permite colocar imagens no documento

\title{ \textbf{Lista 6 - MAC105}}
\date{}
\author{ \textbf{João Gabriel Basi - $\text{N}^\circ$ USP: 9793801}}
\begin{document}
\maketitle
\begin{enumerate}
\item[1.]
\begin{enumerate}
\item[(a)]
Partindo da implicação (1)$a \equiv b \Rightarrow a \simeq b$ (ou seja, $\equiv$ é um refinamento de $\simeq$, já que $\equiv \, \subseteq \, \simeq$) se, para todo $a, b\in A$, $a\equiv b \Leftrightarrow a,b\in [a]_{\equiv}$ e $a\simeq b \Leftrightarrow a,b\in [a]_{\simeq}$ podemos substituir na implicação (1) e teremos que $a,b\in [a]_{\equiv} \Rightarrow a,b\in [a]_{\simeq}$. Com isso, podemos deduzir que $[a]_\equiv \subseteq [a]_\simeq$.
\item[(b)]
Se $a \equiv b \Rightarrow a \simeq b$, sempre haverá uma classe $[a]_\simeq$ correspondente para cada classe $[a]_\equiv$, com $a \in A$, pois $\equiv$ é um refinamento de $\simeq$. Além disso, nunca haverá mais de uma classe em $\simeq$ correspondente à mesma classe de $\equiv$, pois ambas são relações de equivalência, concluindo que a função está bem definida.
\item[(c)]
Se $m|n$, então, quando $a\; mod\; n$ atinge seus valores mínimo e máximo, $a\; mod\; m$ também atinge, sincronizando os valores dos módulos, pois $m$ é múltiplo de $n$. Já quando $m$ não divide $n$, isso não ocorre, então um mesmo valor em $\mathbb{Z}_n$ pode ter mais de um valor em $\mathbb{Z}_m$. Exemplificando, temos:
\begin{center}
\begin{tabular}{ | c | c | c | c |} \hline 
$\mathbb{Z}$ & $\mathbb{Z}_n = \mathbb{Z}_6$ & $\mathbb{Z}_m = \mathbb{Z}_3$ & $\mathbb{Z}_m = \mathbb{Z}_4$\\ \hline
0 & 0 & 0 & 0\\ \hline
1 & 1 & 1 & 1\\ \hline
2 & 2 & 2 & 2\\ \hline
3 & 3 & 0 & 3\\ \hline
4 & 4 & 1 & 0\\ \hline
5 & 5 & 2 & 1\\ \hline
6 & 0 & 0 & 2\\ \hline
\end{tabular}
\end{center}
Podemos ver que $f(0\; mod\; 6) = f(0) = 0\; mod\; 3 = 0\; mod\; 4 = 0$, porém, para $f(6\; mod\; 6) = f(0) = 6\; mod\; 3 = 0$ mas $f(6\; mod\; 6) = f(0) = 7\; mod\; 4 = 3$ contradizendo o valor anterior de $f(0) = 0\; mod\; 4 = 0$. Concluindo que a função não está bem definida se $m$ não divide $n$, mas está bem definida se $m|n$.
\end{enumerate}
\item[2.]
Manipulando um pouco a combinação temos $\binom{p}{i} = \frac{p!}{i!(p-i)!} = p\frac{(p-1)!}{i!(p-i)!}$. Como $\binom{p}{i}$ tem que ser inteiro e $p$ não tem mais divisores além de 1 (sendo que $\frac{p}{1}=p$) e $p$ (que não tem como aparecer no denominador pois $i \leqslant p-1$), o valor de $\frac{(p-1)!}{i!(p-i)!} = k$ sempre será um inteiro para qualquer $p$ primo, então como $\binom{p}{i} = pk$, $p|\binom{p}{i}$, então $\binom{p}{i} \equiv 0\; mod\; p$, concluindo a demonstração.
\item[5.]
Para $x^2 \equiv x\; mod\; 100$ temos 25 e 76;\\
Para $x^2 \equiv x\; mod\; 1000$ temos 625 e 376; \\
Para $x^2 \equiv x\; mod\; 10000$ temos 9376. \\

\end{enumerate}
\end{document}
