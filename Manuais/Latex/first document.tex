\documentclass[12pt, a4paper]{article} %mostra o tipo do documento
\setlength{\topmargin}{-.5in}
\setlength{\textheight}{9in}
\setlength{\textwidth}{6.3in}
\setlength{\oddsidemargin}{-.125in}
\setlength{\evensidemargin}{-.125in}
\usepackage[brazil]{babel} %permite escrever em português
\usepackage[utf8]{inputenc}
\usepackage[a4paper, textheight=260mm, textwidth=162mm]{geometry} %ajusta as margens
\usepackage[T1]{fontenc} %define a fonte das letras
\usepackage{color} %colore as letras
\usepackage{url} %inclui urls
\usepackage[pdfencoding=unicode]{hyperref} %transforma links em texto comum para clicar
\usepackage{amsmath, amssymb, amsthm, amsfonts} %permite fazer textos matemáticos
\usepackage{float} % permite mover tabelas e figuras para qualquer ponto da página
\usepackage{graphicx} %permite colocar imagens no documento
\usepackage{xspace} %coloca em espaço matemático automaticamente
\usepackage{pifont}
\usepackage{fancybox} %adiciona uma box
\usepackage{indentfirst}
\newcommand{\bigmc}[3]{$x=\dfrac{- #2 \pm\sqrt{#2^2-4\cdot#1\cdot#3}}{2\cdot#1}$}
\pagestyle{empty}
\newcommand{\R}{\ensuremath{\mathbb{R}}\xspace}
\newcommand{\Q}{\ensuremath{\mathbb{Q}}\xspace}
\newcommand{\Z}{\ensuremath{\mathbb{Z}}\xspace}
\renewcommand{\C}{\ensuremath{\mathbb{C}}\xspace}
\newcommand{\CC}{\ensuremath{\mathcal{C}}\xspace}
\newcommand{\serio}{\ding{98}\xspace}
\newcommand{\A}{\ensuremath{\mathcal{A}}\xspace}
\renewcommand{\L}{\ensuremath{\mathcal{L}}\xspace}
\newcommand{\tq}{\,|\,}
\newcommand{\val}[1]{\serio\ovalbox{\textbf{#1}}}
\newcommand{\conj}[2]{\ensuremath{\{#1\,|\;#2\}}}
\newcommand{\bl}[1]{\textcolor{red}{\mathbf{#1}}}
%cria um comando novo que executa uma função {\<nome do comando>}[numero de parametros]{"função" utilizando #1, #2... para se referir aos parametros}
\DeclareMathOperator{\seno}{sen}


\title{My first \LaTeX{} document}
\date{16 de Março de 2016}
\author{João Gabriel Bási 02}
\begin{document}
\maketitle

Hello \textbf{world}, \textit{here} is {\color{red} my} \texttt{first} \underline{document!} \\
% Comment
\tiny tiny \scriptsize scriptsize \small small \normalsize normalsize \large large \Large Large \LARGE LARGE \huge huge \Huge Huge \normalsize
%tamanhos das fontes

\begin{enumerate}
\item number 1
\begin{enumerate}
\item subitem number 1.1
\end{enumerate}
\item number 2
\item[42.] the answer
\end{enumerate}

Subscript $K_i$ or $E_{el}$, superscript $x^2$ or $x^{12}$ and both $x^{10}_o$.

\begin{align}
0 &= 3x^3+3x^2-6x   \\
   &= 3(x^3+x^2-2x)   \\
   &= 3x(x^2+x-2)        \\
   &= 3x(x+2)(x-1)        \\
   &\therefore              \\
x &= 0, -2, 1\\
\end{align}
%colocar {align*} se não quiser os números na direita
%& alinha o que tiver embaixo nesse ponto e \\ quebra a linha
bmatrix:
$
\begin{bmatrix}
1 & 2 & 3 \\
a & b & c \\
\end{bmatrix}
\text { pmatrix: }
\begin{pmatrix}
1 & 2 & 3 \\
a & b & c \\
\end{pmatrix}
\text { vmatrix: }
\begin{vmatrix}
1 & 2 & 3 \\
a & b & c \\
\end{vmatrix}
$ \\

Some fractions: $\dfrac{x^2}{x^3} = \frac{1}{x^2}$, an square root: $\sqrt[3]{15}$, an highlighted integral: $$\int_{a}^{b}f(x)\pm \sqrt{50} dx$$ and now a sin: $\seno(\frac{\pi}{2}) = 1\text{ ou sen}(\frac{\pi}{2})$ \\
Alguns símbolos:
$
\forall x, \exists y\in\mathbb{Z} : x \geqslant y \text{ and } x \leqslant y^2 \\
\forall z, \nexists y \in R\cap T : {(x,y)}  \subseteq R\cup T \\
\forall \delta, \exists \varepsilon \leqslant |\delta - x| \rightarrow +\infty \Longleftrightarrow \varepsilon > 0 \\
\bar{x} \quad \vec{x} \quad x^\prime \quad \neg \quad \preceq \quad \not \preceq \quad \not =
\quad \equiv \quad \simeq \quad \gg \quad \ll \quad \sim$ \\
$\displaystyle \lim_{x\rightarrow0}\cos x = 1$ %o "x -> 0" só aparecerá embaixo do limite se ele for colocado no modo display ($$$$) ou no \displaystyle, se não ele aparece como um subscrip 
$$\boxed{\sum_{i=0}^n = \begin{pmatrix} i \\ n \end{pmatrix} a^{n-i}b^i}$$ \\

\begin{center}
\begin{tabular}{ | l | l | l | p{5cm} |} \hline 
Dia& Temp. Mín.  & Temp. Máx & Resumo\\ \hline
Segunda & 22$^\circ$C   & 28$^\circ$C  & Dia nublado com diversas pancadas de chuva;\\ \hline
Terça       & 25$^\circ$C   & 32$^\circ$C  & Manhã nublada abrindo o sol à tarde.  Pequenos períodos de chuvas esparsas pela manhã;\\ \hline
Quarta    & 28$^\circ$C   & 37$^\circ$C  & Dia de sol com pancadas de chuva ao anoitecer; \\ \hline
\end{tabular}
\end{center}

\href{http://www.ctan.org/tex-archive/info/lshort/english/lshort.pdf}{LaTeX (manual)}\\
\bigmc{1}{2}{3}

\includegraphics[scale=0.5]{latex.png}
\end{document}