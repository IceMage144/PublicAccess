\documentclass[12pt, a4paper]{article} %mostra o tipo do documento
\setlength{\topmargin}{-.5in}
\setlength{\textheight}{9in}
\setlength{\textwidth}{6.3in}
\setlength{\oddsidemargin}{-.125in}
\setlength{\evensidemargin}{-.125in}
\usepackage[brazil]{babel} %permite escrever em português
\usepackage[utf8]{inputenc}
\usepackage[a4paper, textheight=260mm, textwidth=162mm]{geometry} %ajusta as margens
\usepackage[T1]{fontenc} %define a fonte das letras
\usepackage{color} %colore as letras
\usepackage{url} %inclui urls
\usepackage[pdfencoding=unicode]{hyperref} %transforma links em texto comum para clicar
\usepackage{amsmath, amssymb, amsthm, amsfonts} %permite fazer textos matemáticos
\usepackage{float} % permite mover tabelas e figuras para qualquer ponto da página
\usepackage{graphicx} %permite colocar imagens no documento
\usepackage{xspace} %coloca em espaço matemático automaticamente


\title{Manual}
\author{João Gabriel Basi}
\begin{document}
\maketitle
\Large
\begin{enumerate}
\item \textbf{Python}
\normalsize
\begin{enumerate}
\item[1.1.] \textbf{Indentação} 
\begin{itemize}
\item A indentação é feita por espaços, enter e dois pontos
\end{itemize}
\item[1.2.] \textbf{Variáveis} 
\begin{itemize}
\item Os tipos das variáveis em Python são determinadas pelo compilador, mas podem ser declarados ao declarar a variável
\item Strings são sempre consideradas como listas de caracteres
\end{itemize}
\item[1.3.] \textbf{Listas}
\begin{itemize}
\item Podem mesclar tipos de variáveis
\item São declaradas como: 
\begin{itemize}
\item lista1 = [] 
\item lista2 = [0]*5  (para 5 posições com zero)
\item lista3 = [0 for i in range(5)] (mesmo do de cima)
\item lista4 = [0, 0, 0, 0, 0]
\end{itemize}
\item Podem ser do tipo array, usando [] (os valores podem ser alterados à vontade), tupla, usando () (os valores não podem ser alterados) ou set, usando \{\} (não pode ter valores repetidos) 
\item São referenciados como: \\
list[0] = 0
\end{itemize}
\item[1.4.] \textbf{Funções}
\begin{itemize}
\item Tipos dos argumentos não precisam ser fornecidos
\item São declaradas como: \\
def area(altura, largura):\\[-0.5cm]

\setlength{\parindent}{1cm} return altura * largura
\end{itemize}
\item[1.5.] \textbf{Objetos}

\item[1.6.] \textbf{Expressões}
\begin{enumerate}
\item[1.6.1.] \textbf{Condicionais}
\begin{itemize}
\item if
\item elif
\item else
\end{itemize}
\item[1.6.2.] \textbf{Loops}
\begin{itemize}
\item while (normal)
\item O for é escrito de outro jeito:\\
for i in range($<inicio>, <fim+1>, <passo>$):\\[-0.5cm]

\setlength{\parindent}{1cm} (O que cada loop faz)
\end{itemize}
\item[1.6.3.] \textbf{Outros}
\begin{itemize}
\item 
\end{itemize}
\end{enumerate}
\item[1.7.] \textbf{Comparadores e booleanos}
\begin{itemize}
\item $==$ e !$=$
\item $<=$ e $>=$
\item $<$ e $>$
\item $\&\&$ ou and
\item $||$ ou or
\item  ! ou not
\end{itemize}
\item[1.8.] \textbf{Operadores}
\begin{itemize}
\item $+$ e $-$
\item * e /
\item ** (potenciação)
\item \% (módulo)
\item // (divisão inteira)
\end{itemize}
\item[1.9.] \textbf{Bibliotecas}
\begin{itemize}
\item São importadas de um dos modos:
\begin{itemize}
\item import <nome>
\item import <nome> as <novo nome>
\item from <nome> import <função> (para importar uma ou mais funções sem precisar de referência)
\item from <nome> import * (para importar todas as funções sem precisar de referência)
\end{itemize}
\item Mais usadas:
\begin{itemize}
\item matplotlib.pyplot: gráficos
\item numpy ou math: matemática
\item random: função random
\end{itemize}
\end{itemize}
\item[1.10.] \textbf{Input e output}
\begin{itemize}
\item Input: input() (retorna uma string, então tem que ser transformado para o tipo desejado)
\item Output: print()
\end{itemize}
\item[1.11.] \textbf{Comentários}
\begin{itemize}
\item Só podem ser feitos de linha em linha com $\#$
\end{itemize}
\item[1.12.] \textbf{Funções e métodos úteis}
\begin{itemize}
\item len(list)
\item Métodos das listas, tuplas e sets
\end{itemize}
\end{enumerate}
%===================================================
\Large
\item \textbf{C e C++}
\normalsize
\begin{enumerate}
\item[1.1.] \textbf{Indentação} 
\begin{itemize}
\item A indentação é feita por chaves e ponto e vírgula
\end{itemize}
\item[1.2.] \textbf{Variáveis} 
\begin{itemize}
\item Os tipos das variáveis têm que ser especificados ao serem criadas, segundo a tabela:
\begin{center}
\begin{tabular}{ | p{4cm} | l | l | l |} \hline 
Tipo & Declaração & Referência & Bits \\ \hline
inteiro & (unsigned) int & \% d & 32 bits \\ \hline
inteiro longo& (unsigned) long long & \% lld & 64 bits\\ \hline
ponto flutuante & (unsigned) float & \% f & ?? \\ \hline
ponto flutuante de dupla precisão & (unsigned) double & \% g & ?? \\ \hline
caractere & char & \% c ou \% s (strings) & 16 bits \\ \hline
ponteiro & <tipo> * & \% p & H \\ \hline
nada & void & --- & --- \\ \hline
\end{tabular}
\end{center}
\item Em C as stringas são criadas como listas de caracteres
\item Especiais C$++$:
\begin{center}
\begin{tabular}{ | l | l | l | l |} \hline 
Tipo & Declaração & Referência & Bits \\ \hline
string & string & \% s & ?? \\ \hline
booleana & bool & ?? & 1 bit\\ \hline
\end{tabular}
\end{center}
\end{itemize}
\item[1.3.] \textbf{Listas}
\begin{itemize}
\item Seus itens só podem ser do tipo declarado
\item São declaradas como: 
\begin{itemize}
\item <tipo da lista> lista1[<espaços>]; 
\item int lista2[] = \{1, 3, 5, 7\};
\end{itemize} 
\item São referenciados como: \\
list[0] = 0
\end{itemize}
\item[1.4.] \textbf{Funções}
\begin{itemize}
\item Tipos dos argumentos precisam ser fornecidos e o tipo do valor de retorno também (o tipo especificado antes do nome da função), se a função não tiver retorno, colocar void
\item São declaradas como: \\
int area(int altura, int largura)\{ \\[-0.5cm]

\setlength{\parindent}{1cm} return (altura * largura);\\
\}
\end{itemize}
\item[1.5.] \textbf{Objetos}
\begin{itemize}
\item Podem ser class (variáveis automaticamente em privado) ou struct (variáveis automaticamente em público)
\item São declarados:
class Rectangle \{ \\[-0.5cm]

\setlength{\parindent}{1cm} int altura; \\[-0.5cm]

\setlength{\parindent}{1cm} int largura; \\[-0.5cm]

\setlength{\parindent}{1cm} public: \\[-0.5cm]

\setlength{\parindent}{2cm} void setDim (int alt, int lar)\{ \\[-0.5cm]

\setlength{\parindent}{3cm} altura = alt; \\[-0.5cm]

\setlength{\parindent}{3cm} largura = lar;\} \\[-0.5cm]

\setlength{\parindent}{2cm} int getDim()\{ \\[-0.5cm]

\setlength{\parindent}{3cm} return \{altura, largura\};\} \\
\}
\end{itemize}
\item[1.6.] \textbf{Expressões}
\begin{enumerate}
\item[1.6.1.] \textbf{Condicionais}
\begin{itemize}
\item if
\item else if
\item else
\end{itemize}
\item[1.6.2.] \textbf{Loops}
\begin{itemize}
\item while
\item for
\item do\{(alguma coisa)\}while(<condição>);
\end{itemize}
\item[1.6.3.] \textbf{Outros}
\begin{itemize}
\item switch
\end{itemize}
\end{enumerate}
\item[1.7.] \textbf{Comparadores e booleanos}
\begin{itemize}
\item $==$ e !$=$
\item $<=$ e $>=$
\item $<$ e $>$
\item $\&\&$
\item $||$
\item  !
\end{itemize}
\item[1.8.] \textbf{Operadores}
\begin{itemize}
\item Decimais:
\begin{itemize}
\item $+$ e $-$
\item * e /
\item \% (módulo)
\end{itemize}
\item Binários:
\begin{itemize}
\item $\&$ (and)
\item | (or)
\item $\sim$ (not)
\item $\wedge$ (xor)
\item $<<$ (left shift)
\item $>>$ (right shift)
\end{itemize}
\end{itemize}
\item[1.9.] \textbf{Bibliotecas}
\begin{itemize}
\item São importadas do modo:
\begin{itemize}
\item \#include $<<$nome$>>$
\end{itemize}
\item Mais usadas:
\begin{itemize}
\item <stdio.h> : inputs e outputs para C
\item <iostream> : inputs e outputs para C$++$
\item <math.h> : matemática
\item <string.h> : strings
\item <vector> : vetor fácil
\item <map> : mapa
\item <queue> : fila fácil
\item <stack> : pilha fácil
\end{itemize}
\end{itemize}
\item[1.10.] \textbf{Input e output}
\begin{itemize}
\item Input:
\begin{itemize}
\item stdio.h: scanf("<referência>", \&<variável>); (se a variável for uma string, não colocar o \&)
\item iostream: cin $>>$ <variável>;
\end{itemize}
\item Output:
\begin{itemize}
\item stdio.h: printf("<referência> e frase", <variável>);
\item iostream: cout $<<$ <variável> $<<$ endl;
\end{itemize}
\end{itemize}
\item[1.11.] \textbf{Comentários}
\begin{itemize}
\item Podem ser feitos em uma linha com //, ou em várias com /* */
\end{itemize}
\item[1.12.] \textbf{Funções e métodos úteis}
\begin{itemize}
\item pow(base, expoente) (potenciação) (incluída no math.h)
\item string.size() (acha o tamanho da string) (incuída no string.h)
\item olhar site cplusplus
\end{itemize}
\end{enumerate}
%========================================================
\Large
\item \textbf{JavaScript}
\normalsize
\begin{enumerate}
\item[1.1.] \textbf{Indentação} 
\begin{itemize}
\item A indentação é feita por chaves e ponto e vírgula
\end{itemize}
\item[1.2.] \textbf{Variáveis} 
\begin{itemize}
\item O tipo da variável não precisa ser especificado mas ela precisa ser declarada usando:\\
var variável = 42;
\end{itemize}
\item[1.3.] \textbf{Listas}
\begin{itemize}
\item Podem mesclar tipos de variáveis
\item São declaradas como: 
\begin{itemize}
\item lista1 = [] ;
\item lista2 = [0, 0, 0, 0, 0];
\item lista3 = new Array();
\end{itemize}
\item São referenciados como: \\
list[0] = 0;
\end{itemize}
\item[1.4.] \textbf{Funções}
\begin{itemize}
\item Tipos dos argumentos não precisam ser especificados
\item São declaradas como: \\
var area = function(altura, largura)\{ \\[-0.5cm]

\setlength{\parindent}{1cm} return (altura * largura);\\
\}
\item Se a função for uma nova classe de objeto, ela deve ser declarada como:
function Rectangle\{ \\[-0.5cm]

\setlength{\parindent}{1cm} this.altura = altura; (variável publica)\\[-0.5cm]

\setlength{\parindent}{1cm} this.largura = largura;\\[-0.5cm]

\setlength{\parindent}{1cm} var cor = "branco"; (variável privada)\\
\}
\item Se já existir uma classe e é preciso adicionar algo a mais, usar:\\
Rectangle.prototype.area = this.altura * this.largura;
\item Se uma classe é uma subclasse de outra, podemos importar os métodos de uma para a outra usando:\\
Square.prototype = new Rectangle;
\end{itemize}
\item[1.5.] \textbf{Objetos}
\begin{itemize}
\item São declarados:\\
var Rectangle = \{ \\[-0.5cm]

\setlength{\parindent}{1cm} altura: 30, \\[-0.5cm]

\setlength{\parindent}{1cm} largura: 40 \\
\}
\item Ou, podem ser declarados:\\
var Rectangle = new Object(); \\
Rectangle.altura = 30; \\
Rectangle.largura = 40; \\
\end{itemize}
\item[1.6.] \textbf{Expressões}
\begin{enumerate}
\item[1.6.1.] \textbf{Condicionais}
\begin{itemize}
\item if
\item else if
\item else
\end{itemize}
\item[1.6.2.] \textbf{Loops}
\begin{itemize}
\item while
\item do-while
\item for
\item for-in
\end{itemize}
\item[1.6.3.] \textbf{Outros}
\begin{itemize}
\item switch
\end{itemize}
\end{enumerate}
\item[1.7.] \textbf{Comparadores e booleanos}
\begin{itemize}
\item $===$ e !$==$
\item $<=$ e $>=$
\item $<$ e $>$
\item $\&\&$
\item $||$
\item  !
\end{itemize}
\item[1.8.] \textbf{Operadores}
\begin{itemize}
\item $+$ e $-$
\item * e /
\item \% (módulo)
\end{itemize}
\item[1.9.] \textbf{Bibliotecas}
\begin{itemize}
\item Mais usadas:
\begin{itemize}
\item Math : matemática
\end{itemize}
\end{itemize}
\item[1.10.] \textbf{Input e output}
\begin{itemize}
\item Input:
\begin{itemize}
\item prompt(<pergunta>, <resposta padrão>); (retorna a resposta do usuário)
\item confirm(<pergunta>); (retorna true ou false dependendo do botão apertado)
\end{itemize}
\item Output:
\begin{itemize}
\item console.log(<frase>); (console)
\item alert(<frase>); (pop-up)
\end{itemize}
\end{itemize}
\item[1.11.] \textbf{Comentários}
\begin{itemize}
\item Podem ser feitos somente em uma linha com //
\end{itemize}
\item[1.12.] \textbf{Funções e métodos úteis}
\begin{itemize}
\item Math.random()
\item list.length
\item string.toUpperCase() e string.toLowerCase()
\item object.hasOwnProerty("prop") (verifica se o objeto tem uma propriedade chamada "prop")
\item num.toFixed(dig) (limita o número de casas decimais de um "num" float em "dig")
\item string.substring(<começo>,<final>); (retorna a substring)
\end{itemize}
\end{enumerate}
%===============================
\Large
\item \textbf{HTML e CSS}
\normalsize
\begin{enumerate}
\item[1.1.] \textbf{Tags} 
\begin{itemize}
\item As tags são a forma de identificação dos comandos e são feitas por <(comando)> (texto) </(comando)>
\item As tags tem atributos que são colocados na tag de abertura, como: <p style="font-size: 10px; color: red">(texto vermelho com fonte tamanho 10)</p>
\end{itemize}
\item[1.2.] \textbf{Layout}
\begin{itemize}
\item Sempre começar o documento com <!DOCTYPE html> e escrever o código entre <html> e </html>
\item Utilizar <body> para o que vai na página e <head> para títulos (body tem atributo style)
\item Em <head> colocar <title> para a frase que vai escrita na guia
\end{itemize}
\item[1.3.] \textbf{Edição de texto}
\begin{itemize}
\item Título: Pode ser feito de vários tamanhos com <h1> (maior) até <h6> (menor)
\item Negrito : <strong>
\item Itálico: <em>
\item Cor da letra : atributo style="color: (cor)"
\item Cor de fundo: atrubuto style="background-color: (cor)"
\item Local do texto: atributo style="text-align:(right, center ou left)"
\item Tamanho da fonte: atributo style="font-size: XXpx" (XX é o tamanho da fonte em pixels)
\item Estilo da fonte: atributo style="font-family: (nome)"
\item Ver \url{https://www.w3.org/TR/CSS21/fonts.html#generic-font-families}
\end{itemize}
\item[1.4.] \textbf{Indentação de frases}
\begin{itemize}
\item Um novo parágrafo é feito por <p>(frase)</p>
\item Uma lista numerada é feita por:\\
<ol>\\
<li>(primeiro item)</li>\\
<li>(segundo item)</li>\\
<li>(terceiro item)</li>\\
...\\
</ol>
\item Uma lista de itens é feita do mesmo jeito da numerada mas substituindo <ol> por <ul>
\item Todos tem o atributo style
\end{itemize}
\item[1.5.] \textbf{Especiais}
\begin{itemize}
\item Imagem: <img src="(link)" /> (não tem fechamento)
\item Hiperlink: <a href=(link)>(frase)</a>
\end{itemize}
\item[1.2.] \textbf{Variáveis} 
\begin{itemize}
\item O tipo da variável não precisa ser especificado mas ela precisa ser declarada usando:\\
var variável = 42;
\end{itemize}
\item[1.3.] \textbf{Listas}
\begin{itemize}
\item Podem mesclar tipos de variáveis
\item São declaradas como: 
\begin{itemize}
\item lista1 = [] ;
\item lista2 = [0, 0, 0, 0, 0];
\item lista3 = new Array();
\end{itemize}
\item São referenciados como: \\
list[0] = 0;
\end{itemize}
\item[1.4.] \textbf{Funções}
\begin{itemize}
\item Tipos dos argumentos não precisam ser especificados
\item São declaradas como: \\
var area = function(altura, largura)\{ \\[-0.5cm]

\setlength{\parindent}{1cm} return (altura * largura);\\
\}
\item Se a função for uma nova classe de objeto, ela deve ser declarada como:
function Rectangle\{ \\[-0.5cm]

\setlength{\parindent}{1cm} this.altura = altura; (variável publica)\\[-0.5cm]

\setlength{\parindent}{1cm} this.largura = largura;\\[-0.5cm]

\setlength{\parindent}{1cm} var cor = "branco"; (variável privada)\\
\}
\item Se já existir uma classe e é preciso adicionar algo a mais, usar:\\
Rectangle.prototype.area = this.altura * this.largura;
\item Se uma classe é uma subclasse de outra, podemos importar os métodos de uma para a outra usando:\\
Square.prototype = new Rectangle;
\end{itemize}
\item[1.5.] \textbf{Objetos}
\begin{itemize}
\item São declarados:\\
var Rectangle = \{ \\[-0.5cm]

\setlength{\parindent}{1cm} altura: 30, \\[-0.5cm]

\setlength{\parindent}{1cm} largura: 40 \\
\}
\item Ou, podem ser declarados:\\
var Rectangle = new Object(); \\
Rectangle.altura = 30; \\
Rectangle.largura = 40; \\
\end{itemize}
\item[1.6.] \textbf{Expressões}
\begin{enumerate}
\item[1.6.1.] \textbf{Condicionais}
\begin{itemize}
\item if
\item else if
\item else
\end{itemize}
\item[1.6.2.] \textbf{Loops}
\begin{itemize}
\item while
\item do-while
\item for
\item for-in
\end{itemize}
\item[1.6.3.] \textbf{Outros}
\begin{itemize}
\item switch
\end{itemize}
\end{enumerate}
\item[1.7.] \textbf{Comparadores e booleanos}
\begin{itemize}
\item $===$ e !$==$
\item $<=$ e $>=$
\item $<$ e $>$
\item $\&\&$
\item $||$
\item  !
\end{itemize}
\item[1.8.] \textbf{Operadores}
\begin{itemize}
\item $+$ e $-$
\item * e /
\item \% (módulo)
\end{itemize}
\item[1.9.] \textbf{Bibliotecas}
\begin{itemize}
\item Mais usadas:
\begin{itemize}
\item Math : matemática
\end{itemize}
\end{itemize}
\item[1.10.] \textbf{Input e output}
\begin{itemize}
\item Input:
\begin{itemize}
\item prompt(<pergunta>, <resposta padrão>); (retorna a resposta do usuário)
\item confirm(<pergunta>); (retorna true ou false dependendo do botão apertado)
\end{itemize}
\item Output:
\begin{itemize}
\item console.log(<frase>); (console)
\item alert(<frase>); (pop-up)
\end{itemize}
\end{itemize}
\item[1.11.] \textbf{Comentários}
\begin{itemize}
\item Podem ser feitos em quantas linhas for preciso com <!-- (comentário) -->
\end{itemize}
\item[1.12.] \textbf{Funções e métodos úteis}
\begin{itemize}
\item Math.random()
\item list.length
\item string.toUpperCase() e string.toLowerCase()
\item object.hasOwnProerty("prop") (verifica se o objeto tem uma propriedade chamada "prop")
\item num.toFixed(dig) (limita o número de casas decimais de um "num" float em "dig")
\item string.substring(<começo>,<final>); (retorna a substring)
\end{itemize}
\end{enumerate}
\end{enumerate}
\end{document}