\documentclass[12pt, a4paper]{article} %mostra o tipo do documento
\setlength{\topmargin}{-.5in}
\setlength{\textheight}{9in}
\setlength{\textwidth}{6.3in}
\setlength{\oddsidemargin}{-.125in}
\setlength{\evensidemargin}{-.125in}
\usepackage[brazil]{babel} %permite escrever em português
\usepackage[utf8]{inputenc}
\usepackage[a4paper, textheight=260mm, textwidth=162mm]{geometry} %ajusta as margens
\usepackage[T1]{fontenc} %define a fonte das letras
\usepackage{amsmath, amssymb, amsthm, amsfonts} %permite fazer textos matemáticos

\title{Relatório EP1 - MAC0121}
\date{}
\author{João Gabriel Basi - $\text{N}^\circ$ USP: 9793801}
\begin{document}
\maketitle
\begin{enumerate}
\large
\item[1.]\textbf{Conceitos matemáticos e simplificações utilizados}
\normalsize\\
Chamando a função de Collatz de $f$, a função que determina o número de passos de $p$ e o intervalo especificado na entrada do programa de $[i,j]$:
\begin{itemize}
\item A partir de um número inicial $a\in[i,j]$, obti os valores de $p$ para todos os inteiros $b_x$ (sendo $x$ o número de iterações de $f$ necessárias para obter $b_x$ a partir de $a$) encontrados a cada iteração da função $f$, pela fórmula $p(b_x) = p(a) - x$.
\item Se um número $b_x$ obtido a partir de $a$ já tiver o seu valor de $p$ guardado no vetor, utilizei a fórmula $p(a) = p(b_x) + x$ para obter o valor de $p(a)$.
%\item Encontrei o valor de $p$ para todos os inteiros $c_y \leqslant j$ da forma $a\cdot2^y$ pela fórmula $p(c_y) = p(a) + y$ (sendo $y$ um inteiro positivo) utilizando a regra para obter números pares da função inversa $f^{-1}$, $y$ vezes.
\end{itemize}
\large
\item[2.]\textbf{Observações sobre a função}
\normalsize\\
Ainda utilizando as variáveis e funções definidas no item anterior:
\begin{itemize}
\item Quanto maior o $a$, maior é a chance de números consecutivos a ele terem o mesmo valor de $p$.
\item Se utilizarmos a função $f^{-1}$ a partir do 1, há vezes em que há dois resultados possíveis, um ímpar (utilizando $f^{-1}(a) = (a-1)/3$) e outro par (utilizando $f^{-1}(a) = 2a$), e há vezes em que só há o resultado par; porém, ao atingir um múltiplo de $3$, passa a ser impossível achar um resultado ímapar, já que não existe $k$ inteiro tal que $f^{-1}(3k) = ((3k)-1)/3$, e a função passa a obter só resultados pares.
\end{itemize}
\large
\item[3.]\textbf{Maior intervalo testado para o código}
\normalsize\\
%O programa trabalha com uma lista de $10^7$ posições, então a diferença $j-i$ tem que ser $\leqslant 10^7$, porém os números testados não correspondem às suas respectivas posições no vetor, e sim à uma posição defina pelo programa, permitindo $i,j \geqslant 10^7$, ainda respeitando a desigualdade $j-i \leqslant 10^7$. Foi testado que, para a grande maioria dos números menores que $10^{11}$, os valores do conjunto $[i,j]$ podem ser calculados sem que haja problema de overflow. Resumindo: as entradas do programa devem ser $i,j \leqslant 10^{11}$, com $j-i \leqslant 10^7$.
Consegui testar até 113382, depois disso alguns números começam a dar overflow no int em alguma das iterações da função
\end{enumerate}
\end{document}