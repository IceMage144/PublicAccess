\documentclass[12pt, a4paper]{article} %mostra o tipo do documento
\setlength{\topmargin}{-.5in}
\setlength{\textheight}{9in}
\setlength{\textwidth}{6.3in}
\setlength{\oddsidemargin}{-.125in}
\setlength{\evensidemargin}{-.125in}
\usepackage[brazil]{babel} %permite escrever em português
\usepackage[utf8]{inputenc}
\usepackage[a4paper, textheight=260mm, textwidth=162mm]{geometry} %ajusta as margens
\usepackage[T1]{fontenc} %define a fonte das letras
\usepackage{color} %colore as letras
\usepackage{url} %inclui urls
\usepackage[pdfencoding=unicode]{hyperref} %transforma links em texto comum para clicar
\usepackage{amsmath, amssymb, amsthm, amsfonts} %permite fazer textos matemáticos
\usepackage{float} % permite mover tabelas e figuras para qualquer ponto da página
\usepackage{graphicx} %permite colocar imagens no documento
\usepackage{indentfirst}
\setlength\parindent{24pt}

\title{Relatório EP3 - MAC0121}
\date{}
\author{João Gabriel Basi - $\text{N}^\circ$ USP: 9793801}
\begin{document}
\maketitle
\section{O programa}
O programa recebe, na linha de comando, um arquivo, um tipo de implemetação de lista ligada e uma ordem de impressão. O programa imprime na saída padrão as todas as palavras do arquivo e suas respectivas ocorrências, tanto em ordem alfabética, ou em ordem de ocorrência, dependendo do que for recebido na linha de comando.

\section{As funções}
As funções foram separadas em bibliotecas de acordo com a sua utilidade:
	\subsection{auxfuncs.h}
		Biblioteca com funções que auxiliam o programa a manusear a memória:
		%struct
		\begin{itemize}
			\item \textit{emalloc}:
			\item \textit{estrdup}:
		\end{itemize}
	\subsection{buffer.h}
		Biblioteca de funções sobre a strutura Buffer:
		%struct
		\begin{itemize}
			\item \textit{BufferCreate}:
			\item \textit{BufferDestroy}:
			\item \textit{BufferReset}:
			\item \textit{BufferPush}:
			\item \textit{readLine}:
		\end{itemize}
	\subsection{vectorfuncs.h}
		Biblioteca com funções comuns entre as tabelas de símbolos implementadas com 		vetor:
		%struct
		\begin{itemize}
			\item \textit{VTableCreate}: 
			\item \textit{VTableDestroy}:
			\item \textit{VTablePush}:
			\item \textit{valcompV}:
			\item \textit{strcompV}:
			\item \textit{mergeSortV}:
		\end{itemize}
	\subsection{linkedlistfuncs.h}
		Biblioteca com funções comuns entre as tabelas de símbolos implementadas com lista ligada:
		%struct
		\begin{itemize}
			\item \textit{LLTableCreate}:
			\item \textit{LLTableDestroy}:
		\end{itemize}
	\subsection{tabelaSimbolo\_VO.h}
		 Biblioteca com funções sobre a tabela de símbolos implementada com vetor ordenado:
		\begin{itemize}
			\item \textit{OVAdd}:
			\item \textit{OVPritVal}:
			\item \textit{OVPrintLexi}:
		\end{itemize}
	\subsection{tabelaSimbolo\_VD.h}
		Biblioteca com funções sobre a tabela de símbolos implementada com vetor desordenado;
		\begin{itemize}
			\item \textit{UVAdd}:
			\item \textit{UVPrintVal}:
			\item \textit{UVPrintLexi}:
		\end{itemize}
	\subsection{tabelaSimbolo\_LO.h}
		Biblioteca com funções sobre a tabela de símbolos implementada com lista ligada ordenada;
		\begin{itemize}
			\item \textit{OLLAdd}:
			\item \textit{OLLPrintVal}:
			\item \textit{OLLPrintLexi}:
		\end{itemize}
	\subsection{tabelaSimbolo\_LD.h}
		Biblioteca com funções sobre a tabela de símbolos implementada com lista ligada desordenada;
		\begin{itemize}
			\item \textit{ULLAdd}:
			\item \textit{ULLPrintVal}:
			\item \textit{ULLPrintLexi}:
		\end{itemize}
	\subsection{tabelaSimbolo\_AB.h}
		Biblioteca com funções sobre a tabela de símbolos implementada com árvore de busca binária.
		%struct
		\begin{itemize}
			\item \textit{BSTTableCreate}:
			\item \textit{BSTTableDestroy}:
			\item \textit{BSTAdd}:
			\item \textit{BSTPrintVal}:
			\item \textit{BSTPrintLexi}:
		\end{itemize}
	\subsection{tabelaSimbolo.c} %Verificar
		\begin{itemize}
			\item \textit{show\_usage}
			\item \textit{executeOV}
			\item \textit{executeUV}
			\item \textit{executeOLL}
			\item \textit{executeULL}
			\item \textit{executeBST}
		\end{itemize}
\end{document}