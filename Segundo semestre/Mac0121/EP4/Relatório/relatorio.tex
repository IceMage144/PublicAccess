\documentclass[12pt, a4paper]{article} %mostra o tipo do documento
\usepackage[brazil]{babel} %permite escrever em português
\usepackage[utf8]{inputenc}
\usepackage[a4paper, textheight=9in, textwidth=6.3in]{geometry} %ajusta as margens
\usepackage[T1]{fontenc} %define a fonte das letras
\usepackage{color} %colore as letras
\usepackage{url} %inclui urls
\usepackage[pdfencoding=unicode]{hyperref} %transforma links em texto comum para clicar
\usepackage{amsmath, amssymb, amsthm, amsfonts} %permite fazer textos matemáticos
\usepackage{float} % permite mover tabelas e figuras para qualquer ponto da página
\usepackage{graphicx} %permite colocar imagens no documento
\usepackage{indentfirst}
\usepackage{multirow}
\setlength\parindent{24pt}
\def\changemargin#1#2{\list{}{\rightmargin#2\leftmargin#1}\item[]}
\let\endchangemargin=\endlist 

\title{Relatório EP3 - MAC0121}
\date{}
\author{João Gabriel Basi - $\text{N}^\circ$ USP: 9793801}
\begin{document}
\maketitle
\section{O programa}
O programa recebe, na linha de comando, um arquivo, um tipo de implemetação de lista ligada e uma ordem de impressão. O programa imprime na saída padrão as todas as palavras do arquivo e suas respectivas ocorrências, tanto em ordem alfabética, ou em ordem de ocorrência, dependendo do que for recebido na linha de comando.

\section{As funções}
	\subsection{No arquivo da main}
	\begin{itemize}
		\item \textit{show\_usage}: Mostra uma mensagem de erro na saída de erro de acordo com o número fornecido;
		\item \textit{executeOV}: Executa o programa utilizando uma tabela de símbolos ordenada e implementada com vetor;
		\item \textit{executeUV}: Executa o programa utilizando uma tabela de símbolos desordenada e implementada com vetor;
		\item \textit{executeOLL}: Executa o programa utilizando uma tabela de símbolos ordenada e implementada com lista ligada;
		\item \textit{executeULL}: Executa o programa utilizando uma tabela de símbolos desordenada e implementada com lista ligada;
		\item \textit{executeBST}: Executa o programa utilizando uma tabela de símbolos implementada com árvore de busca binária;
	\end{itemize}

	\subsection{auxfuncs.h}
	Biblioteca com funções e uma struct que auxiliam o programa a manusear a memória:
	\begin{itemize}
		\item \textit{InsertionResult}: Struct que auxilia no recebimento de dados sobre a inserção de um elemento na tabela de símbolos;
		\item \textit{emalloc}: Aloca um espaço na memória, mostrando uma mensagem de erro se não houver espaço para a alocação;
		\item \textit{estrdup}: Dulica uma string, retornando um ponteiro para a cópia e mostrando uma mensagem de erro se não houver espaço para a alocação;
	\end{itemize}

	\subsection{buffer.h}
	Biblioteca que cria a struct Buffer e cria funções sobre ela:
	\begin{itemize}
		\item \textit{Buffer}: Struct com um vetor de tamanho dinâmico, que auxilia no recebimento de strings da entrada;
		\item \textit{BufferCreate}: Cria um buffer;
		\item \textit{BufferDestroy}: Destroi um buffer;
		\item \textit{BufferReset}: Reseta um buffer;
		\item \textit{BufferPush}: Adiciona um caractere no fim do buffer;
		\item \textit{readLine}: Lê uma linha da entrada e guarda no buffer.
	\end{itemize}

	\subsection{vectorfuncs.h}
	Biblioteca com funções comuns entre as tabelas de símbolos implementadas com 		vetor:
	\begin{itemize}
		\item \textit{Entry}: Struct que guarda uma chave e um valor associado à ela;
		\item \textit{VectorSTable (VST)}: Struct que guarda uma tabela de símbolos com associações string-int por meio de um vetor de Entries;
		\item \textit{VTableCreate}:  Cria uma tabela de símbolos utilizando vetor;
		\item \textit{VTableDestroy}: Destroi uma tabela de símbolos implemetada com vetor;
		\item \textit{VTablePush}: Adiciona uma chave e um valor associado à ela no fim de uma tabela de símbolos feita com vetor;
		\item \textit{valcompV}: Compara dois valores de uma tabela de símbolos implementda com vetor;
		\item \textit{strcompV}: Compara duas chaves de uma tabela de símbolos implementda com vetor;
		\item \textit{mergeSortV}: Organiza uma tabela de símbolos implementada com vetor utilizando a função fornecida.
	\end{itemize}

	\subsection{linkedlistfuncs.h}
	Biblioteca com funções comuns entre as tabelas de símbolos implementadas com lista ligada:
	\begin{itemize}
		\item \textit{LLNode}: Nó para lista ligada com uma chave, um valor associado à ela e um ponteiro para o próxomo nó;
		\item \textit{LinkedListSTable (LLST)}: Cabeça para uma tabela de símbolos com associações string-int feita com uma lista ligada de LLNodes;
		\item \textit{LLTableCreate}: Cria uma tablea de símbolos utilizando lista ligada;
		\item \textit{LLTableDestroy}: Destroi uma tabela de símbolos que utiliza lista ligada.
	\end{itemize}

	\subsection{tabelaSimbolo\_VO.h}
	Biblioteca com funções sobre a tabela de símbolos implementada com vetor ordenado:
	\begin{itemize}
		\item \textit{OVAdd}: Adiciona uma chave à uma tabela de símbolos ordenada e implementada com vetor;
		\item \textit{OVPritVal}: Imprime os elementos de uma tabela de símbolos, ordenada e implementada com vetor, em ordem decrescente de valor;
		\item \textit{OVPrintLexi}: Imprime os elementos de uma tabela de símbolos, ordenada e implementada com vetor, em ordem alfabética; 
	\end{itemize}

	\subsection{tabelaSimbolo\_VD.h}
	Biblioteca com funções sobre a tabela de símbolos implementada com vetor desordenado;
	\begin{itemize}
		\item \textit{UVAdd}: Adiciona uma chave à uma tabela de símbolos desordenada e implementada com vetor;
		\item \textit{UVPrintVal}: Imprime os elementos de uma tabela de símbolos, desordenada e implementada com vetor, em ordem decrescente de valor;
		\item \textit{UVPrintLexi}: Imprime os elementos de uma tabela de símbolos, desordenada e implementada com vetor, em ordem alfabética; 
	\end{itemize}

	\subsection{tabelaSimbolo\_LO.h}
	Biblioteca com funções sobre a tabela de símbolos implementada com lista ligada ordenada;
	\begin{itemize}
		\item \textit{OLLAdd}: Adiciona uma chave à uma tabela de símbolos ordenada e implementada com lista ligada;
		\item \textit{OLLPrintVal}: Imprime os elementos de uma tabela de símbolos, ordenada e implementada com lista ligada, em ordem decrescente de valor;
		\item \textit{OLLPrintLexi}: Imprime os elementos de uma tabela de símbolos, ordenada e implementada com lista ligada, em ordem alfabética; 
	\end{itemize}

	\subsection{tabelaSimbolo\_LD.h}
	Biblioteca com funções sobre a tabela de símbolos implementada com lista ligada desordenada;
	\begin{itemize}
		\item \textit{ULLAdd}: Adiciona uma chave à uma tabela de símbolos desordenada e implementada com lista ligada;
		\item \textit{ULLPrintVal}: Imprime os elementos de uma tabela de símbolos, desordenada e implementada com lista ligada, em ordem decrescente de valor;
		\item \textit{ULLPrintLexi}: Imprime os elementos de uma tabela de símbolos, desordenada e implementada com lista ligada, em ordem alfabética; 
	\end{itemize}

	\subsection{tabelaSimbolo\_AB.h}
	Biblioteca com funções sobre a tabela de símbolos implementada com árvore de busca binária.
	\begin{itemize}
		\item \textit{BTNode}: Nó para árvore binária com uma chave, um valor associado à ela, um ponteiro para o nó direito e um ponteiro para o nó esquerdo;
		\item \textit{BinaryTreeSTable (BTST)}: Raiz para um tabela de símbolos com associações string-int feita com uma árvore binária de BTNodes;
		\item \textit{BSTTableCreate}:  Cria uma tabela de símbolos utilizando árvore binária;
		\item \textit{BSTTableDestroy}: Destroi uma tabela de símbolos implemetada com árvore binária;
		\item \textit{BSTAdd}: Adiciona uma chave à uma tabela de símbolos implementada com árvore de busca binária;
		\item \textit{BSTPrintVal}: Imprime os elementos de uma tabela de símbolos, implementada com árvore de busca binária, em ordem decrescente de valor;
		\item \textit{BSTPrintLexi}: Imprime os elementos de uma tabela de símbolos, implementada com árvore de busca binária, em ordem alfabética; 
	\end{itemize}
\section{Análize dos algoritmos}
	\subsection{Vetor desordenado}
		\subsubsection{Inserção}
		\par A inserção na tabela foi feita de modo linear, comparando a chave a ser inserida com todas as que estão na tabela. Se for achada uma chave igual, a função para de comparar, caso contrário, o algoritmo insere a nova chave no final da lista.
		\par No pior caso, temos que a palavra é inserida no fim da lista, fazendo $n$ comparações.
		\par No caso médio, temos que a média de comparações para uma inserção é:
		\begin{equation}
		E(x) = \sum^n_{i=1}\left(i\cdot \frac{1}{n}\right) = \frac{1}{n}\cdot\sum^n_{i=1}i \stackrel{\text{Soma de P.A.}}{\Longrightarrow} \frac{n(n+1)}{2n} = \frac{n+1}{2}
		\end{equation}
		\par Já em se tratando de movimentação de elementos da lista, o algoritmo de inserção não movimenta nenhum elemento, já que os elementos novos são inseridos no final da lista.
		\subsubsection{Impressão}
		\par Para ambas as ordens de impressão, é preciso ordenar a lista. Para isso, o programa usa um merge sort, que faz $2n\log n$ movimentações e $n\log n$ comparações em todos os casos.

	\subsection{Vetor ordenado}
		\subsubsection{Inserção}
		\par O programa faz uma busca binária para achar o lugar certo de inserção, então move todos os elementos, a partir desse lugar, uma posição à frente, para poder inserir a nova chave. Na busca binária, o algoritmo faz $\log n$ comparações, e colocando as chaves maiores para frente, se considerarmos que a probabilidade de uma palavra entrar em uma lugar da tabela é a mesma para qualquer posição, ele faz, em média, $\frac{n+1}{2}$ movimentações (a equação fica igual à equação (1) do item anterior) e no pior caso, a palavra é inserida no começo da lista, movendo os $n$ elementos para frente.
		\subsubsection{Impressão}
		\par Como o vetor já está ordenado em ordem alfabética, para a impressão em ordem alfabética não são necessários comparações ou movimentações.
		\par Já para a impressão em ordem de frequência, é preciso ordenar o vetor. Para isso foi utilizado um merge sort, que faz $2n\log n$ movimentações e $n\log n$ comparações.

	\subsection{Lista ligada desordenada}
		\subsubsection{Inserção}
		\par O programa faz uma busca linear, que faz, em média, $\frac{n+1}{2}$ comparações por inserção se a chave estiver na lista (assim como calculado na equação (1) da inserção do vetor desordenado), e faz $n$ comparações se ela não estiver; porém não faz nenhuma movimentação, já que o novo nó pode ser inserido trocando o ponteiro do nó anterior.
		\par Se o algoritmo não achar uma chave igual na tabela, ele a insere no final da lista.
		\subsubsection{Impressão}
		\par Como a lista está desordenada, em ambos os casos programa passa as chaves da lista ligada para um vetor, fazendo $n$ movimentações e $n$ comparações, e ordena o vetor com um merge sort, que faz $2n\log n$ movimentações e $n\log n$ comparações, fazendo no total $n(1+2\log n)$ movimentações e $n(1+\log n)$ comparações.

	\subsection{Lista ligada ordenada}
		\subsubsection{Inserção}
		\par O programa faz uma busca linear, que faz, em média, $\frac{n+1}{2}$ comparações (assim como calculado na equação (1) da inserção do vetor desordenado), porém, ele para ao achar uma chave maior ou igual à que será inserida e a insere nessa posição. O pior caso acontece se a nova chave for lexicograficamente maior que todas as da tabela, então o algoritmo faz $n$ comparações (note que o pior caso aqui é mais difícil de acontecer que na lista ligada desordenada, já que aqui a palavra tem que ser lexicograficamente maior que todas as da tabela para ser inserida no final, diferente da desordenada que insere todas as palavras novas no fim da lista).
		\subsubsection{Impressão}
		\par Para a impressão em ordem alfabética, o programa só percorre a lista e imprime seus elementos, fazendo $n$ comparações, já que ela já está ordenada por ordem alfabética. Já para a ordem de ocorrência, o programa faz o mesmo procedimento da impressão da lista ligada desordenada, resultando em $n(1+2\log n)$ movimentações e $n(1+\log n)$ comparações.

	\subsection{Árvore de busca binária}
		\subsubsection{Inserção}
		\par O programa percorre a árvore fazendo, no melhor caso, $\log n$ comparações e, no pior caso, $n$ comparações, porém a inserção nesse local é feita sem comparações ou movimentações, já que só é preciso mudar o ponteiro da folha em que será inserida o novo elemento.
		\par O caso médio é mais complicadode se calcular, pois leva em conta a estrutura do texto utilizado, mas vamos dizer que é muito mais perto de $\log n$ do que $n$, ou seja $\log n + c$.
		\subsubsection{Impressão}
		\par Para a impressão em ordem alfabética, o programa utiliza um algoritmo recursivo que faz $n$ comparações e nenhuma movimentação. Já para a impressão em ordem de ocorrência, o mesmo método da lista ligada desordenada é usado, resultando em $n(1+2\log n)$ movimentações e $n(1+\log n)$ comparações.

\section{Testes}
\par Os testes foram feitos com uma versão do dicionário inglês de 1913 \cite{dict}, em que os caracteres que não estão na tabela ASCII foram retirados, e com uma versão da bíblia em inglês também \cite{bib}. A versão ordenada da bíblia, citada na tabela, foi feita por mim, pegando as palavras da original e ordenando-as. Sendo assim, um arquivo com o mesmo número de palavras do original é gerado, porém as palavras estão organizadas em ordem alfabética. Esse caso é o pior caso de quase todas as tabelas, já que elas adicionam palavras novas e lexicograficamente maiores no final da lista, então isso obriga os algoritmos a checar todas as palavras da lista antes de adicioná-las. A outra versão, a ivertida, é o mesmo da versão ordenada, porém ordenada de "z" à "a", que seria o pior caso do vetor ordenado.\\[-0.5cm]
\begin{changemargin}{-0.7in}{0in}
\begin{center}
\begin{tabular}{| c | c | c | c | c | c | c | c | c |} \hline 
\multicolumn{9}{| c |}{Inserção}\\ \hline
\multirow{2}{*}{Implementação da tabela} & \multicolumn{2}{| c |}{Comparações} & \multicolumn{2}{| c |}{Movimentações} & \multicolumn{4}{| c |}{Tempos}\\ \cline{2-9}
& médio & pior & médio & pior & Dicionário & Bíblia & ordenada & invertida\\ \hline
Vetor desordenado & $\frac{n+1}{2}$ & $n$ & 0 & 0 & 273,37s & 2,7s & 21s & 17s\\ \hline
Vetor ordenado & $\log n$ & $\log n$ & $\frac{n+1}{2}$ & $n$ & 8,72s & 0,2s & 0,16s & 0,23s\\ \hline
Lista ligada desordenada & $\frac{n+1}{2}$ & $n$ & 0 & 0 & 427,5s & 3,3s & 27,2s & 0,12s\\ \hline
Lista ligada ordenada & $\frac{n+1}{2}$ & $n$ & 0 & 0 & 10900s (3h) & 48s & 27,5s & 22,6s\\ \hline
Árvore de busca binária & $\log n + c$ & $n$ & 0 & 0 & 1,78s & 0,2s & 32s & 27,3s\\ \hline 
\end{tabular}
\end{center}
\end{changemargin}
\hfill \\
\hfill
\par Os tempos foram obtidos com a impressão em ordem alfabética. A impressão em ordem de ocorrência não mostrou uma diferença significante nos tempos, por isso não será citada. O tempo de impressão também não é citado pois também não mostrou muita diferença nos tempos.
\par Analizando os resultados, vemos que as maneiras mais eficientes são a árvore de busca binária e o vetor ordenado, e a mais ineficiente é a lista ligada ordenada. Nota-se também que o algoritmo dos dois tipos de lista ligada são bem similares, então eles fazem o mesmo tempo na bíblia ordenada.
\par Outro fato que se pode perceber olhando a tabela, é que a versão ordenada da lista ligada se saiu bem pior que a versão desordenada, o que vai contra a nossa intuição. Fazendo algumas observações, percebi que, na lista ligada desordenada, as palavras mais comuns do vocabulário são adicionadas mais perto da cabeça da lista, dando uma vantagem sobre a lista ligada ordenada, já que palavras como "the" e "them", que são muito comuns no vocabulário inglês, ficam no final da versão ordenada, mas têm muita chance de aparecer no começo da desordenada.
\par Percebe-se também que quase todas as tabelas gastam todo o tempo comparando as chaves, a única em que isso não acontece é a implementada com vetor ordenado, que gasta a maior parte de seu tempo movimentando seu conteúdo para se manter ordenada. Porém essa troca ainda assim melhora sua performance, como é observado nos testes com a bíblia. Assim conseguimos ver que as comparações gastam muito mais tempo que as movimentações.
\begin{thebibliography}{2}
\bibitem{dict} Link para o dicionário \url{http://www.gutenberg.org/ebooks/29765}
\bibitem{bib} Link para a Bíblia \url{http://www.gutenberg.org/ebooks/10}
\end{thebibliography}
\end{document}